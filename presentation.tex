\documentclass[xcolor=dvipsnames]{beamer}
\usepackage{beamerthemelined}
\usepackage{pstricks}

\setbeamertemplate{navigation symbols}{}
\setbeamertemplate{caption}[numbered]

\usepackage{caption}
\captionsetup{font=small}
\captionsetup{labelfont={small,color=blue}}

\title{Investigating an Optical Vortex}
\author{Paho Lurie-Gregg, Grant Sherer, David Grant, Michael Perlin, Aaron Kratzer}
\date{\today}

\usepackage{ifthen,xifthen}
\newenvironment{items}[1][]
{\begin{itemize}
    \ifthenelse{\isempty{#1}}
    {\setlength{\itemsep}{12pt}}{\setlength{\itemsep}{#1}}}
  {\end{itemize}}

%%% Standard math:
\usepackage{amsfonts,amssymb,amsmath,amsthm} % Math packages
\usepackage{braket} % Bra-ket notation stuff
\newcommand{\st}{\displaystyle} % For making small math big
\renewcommand{\t}{\text} % For text in math environment
\renewcommand{\c}{\cdot} % Multiplication dot in math
\newcommand{\f}[2]{\dfrac{#1}{#2}} % Shortcut for fractions
\newcommand{\p}[1]{\left(#1\right)} % Parenthesis
\renewcommand{\sp}[1]{\left[#1\right]} % Square parenthesis
\renewcommand{\set}[1]{\left\{#1\right\}} % Curly parenthesis
\newcommand{\abs}[1]{\left|#1\right|} % Absolute value

%%% Physics symbols, vectors
\renewcommand{\epsilon}{\varepsilon} % Prettier epsilon
\renewcommand{\phi}{\varphi} % Prettier phi
\renewcommand{\l}{\ell} % Prettier l
\renewcommand{\v}[1]{\boldsymbol{\mathrm{#1}}} % Bold vectors
\newcommand{\uv}[1]{\hat{\boldsymbol{\mathrm{#1}}}} % Unit vectors
\newcommand{\del}{\v\nabla} % Del operator
\renewcommand{\d}{\partial} % Partial d
\newcommand{\fd}[2]{\f{d #1}{d #2}} % Derivative
\newcommand{\sd}[2]{\f{d^2 #1}{d^2 #2}} % Derivative
\newcommand{\fpd}[2]{\f{\d #1}{\d #2}} % Partial derivative
\newcommand{\spd}[2]{\f{\d^2 #1}{\d^2 #2}} % Partial derivative

\title{Optical Vortices (THIS BEGS FOR A NEW TITLE!!!111!!!@!@!!!!)}
\author{Aaron Kratzer, David Grant, Paho Lurie-Gregg,
  Michael Perlin, Grant Sherer}
\date{06 December 2013}

\begin{document}

\begin{frame}
  \maketitle
\end{frame}

\begin{frame}
	\frametitle{Generation - Multilayer Spiral Phase Plate}
  \begin{columns}[c]
    \column{2.5in}
    \begin{items}
    \item Electron beam deposition of  SiO$_2$
    \item $\phi$ dependence created through varying thickness of plate
    \end{items}
    \column{2in}
    \begin{figure}
      \includegraphics[width=\columnwidth]{MSPP.jpg}
      \caption{SiO$_2$ Phase Plate}
      \label{MSPP}
    \end{figure}
  \end{columns}
\end{frame}

\begin{frame}
	\frametitle{Generation - Computer Generated Holography}
  \begin{items}
  \item Interference pattern produced using object and reference wave
  \item Filament made from interference pattern
  \item Diffraction occurs as beam goes through filament
  \item First order diffraction beam contains vortex
  \end{items}
\end{frame}

 \begin{frame}
  \centering
    \begin{align*}
  \v A\p{r,\phi,z}&=A_0\f{w_0}{w}\sp{\f{r\sqrt 2}{w}}^\ell
  L^{\p{\ell}}_p\sp{2\f{r^2}{w^2}}\exp\sp{-\f{r^2}{w^2}}
  \exp\sp{-i\f{kr^2z}{2\p{z^2+z_R^2}}} \\
  &~~~~\times\exp\sp{i\p{2p+\ell+1}\arctan\p{\f z{z_R}}}
  \exp\p{i\ell\phi}\exp\p{-i\omega t} \uv x\\
   w&=w_0\sqrt{1+z^2/z_R^2}\\
    L^{\p{\ell}}_p\p{x}
 & =\sum_{m=0}^p\p{-1}^m\f{\p{p+\ell}!}{\p{p-m}!\p{\ell+m}!m!}x^m\\
  z_R&=\f12 kw_0^2
\end{align*}
 \end{frame}

\begin{frame}
\begin{align*}
  L^{\p{3}}_1\p{x}=-x+4\\
  \v E=i\omega\v A &&& \v B=-\del \times\v A
\end{align*}
\end{frame}

\begin{frame}
\begin{align*}
 u&=\f12\p{\epsilon E^2+B^2/\mu} \\
  \v S&=\v E\times\v B/\mu \\
    I&=\f{c\epsilon}2E^2=\f c2\epsilon\omega^2A_I^2=cu/2\\
\end{align*}
\end{frame}

\begin{frame}
\begin{align*}
 \sigma_{ij}&=\epsilon\p{E_iE_j-\f12\delta_{ij}E^2}
  +\f1\mu\p{B_iB_j-\f12\delta_{ij}B^2} \\
    &=\delta_{ij}\f12\p{\epsilon
    E^2+B^2/\mu}=\delta_{ij}u\\
 f_i&=\d_k\sigma_{ik}-\d_tS_i/c^2=\d_iu-\delta_{iz}\d_tu/c
\end{align*}
\end{frame}
% \begin{frame}
%  \begin{figure}[h]
%    \centering
%    \animategraphics[width=\columnwidth, loop]{5}{anim/slice-By-}{00}{14}
%  \end{figure}
% \end{frame}

% \begin{frame}
%  \begin{figure}[h]
%    \centering
%    \animategraphics[width=\columnwidth, loop]{5}{anim/slice-Bz-}{00}{29}
%  \end{figure}
% \end{frame}

% \begin{frame}
%  \begin{figure}[h]
%    \centering
%    \animategraphics[width=\columnwidth, loop]{5}{anim/slice-Sy-}{00}{29}
%  \end{figure}
% \end{frame}

% \begin{frame}
%  \begin{figure}[h]
%    \centering
%    \animategraphics[width=\columnwidth, loop]{5}{anim/slice-Sz-}{00}{29}
%  \end{figure}
% \end{frame}

%---begining of presentation---




%-----Aaron's Application stuff--------
\begin{frame}
	\frametitle{Optical Vortex Application: Twisted Radio Waves}
	\begin{center}
		\emph{Encoding many channels in the same frequency through radio vorticity: first experimental test.}
	\end{center}
	\begin{minipage}{0.49\textwidth}
	\includegraphics[width=\textwidth]{helical_dish}
	\end{minipage}
	\begin{minipage}{0.49\textwidth}
		\begin{items}
		\item Multiple signals transmitted on the same frequency.
		\end{items}
	\end{minipage}

\end{frame}

\end{document}
